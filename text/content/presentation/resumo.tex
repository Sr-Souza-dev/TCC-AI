\chapter*{Resumo}

\noindent 
O presente trabalho propõe o uso de técnicas de IA para prever valores futuros de ações no mercado financeiro. Inicialmente, é apresentada uma contextualização histórica e teórica do mercado de ações, abordando diferentes modelos de análise utilizados e ressaltando a importância da previsão de séries temporais. Feito isso, a principal contribuição do trabalho é o desenvolvimento de uma máquina de previsão baseada em abordagens de Ensemble. Essa máquina recebe múltiplas métricas econômicas como entrada e é capaz de fornecer a probabilidade de um determinado ativo financeiro apresentar um movimento de alta ou baixa. Além disso, foram realizados experimentos iniciais utilizando três algoritmos de classificação (SVM, KNN e LR) em um conjunto de dados do ativo PETR4. Os resultados indicam que o modelo KNN obteve o melhor desempenho, seguido pelo ensemble proposto, enquanto o SVM apresentou resultados semelhantes à estratégia \textit{buy and hold} e o LR teve um desempenho inferior.

\vfill%
\noindent Palavras-chave: Previsão de ações; Mercado financeiro; Inteligência artificial; Séries temporais; Análise de dados financeiros; Ensemble.
