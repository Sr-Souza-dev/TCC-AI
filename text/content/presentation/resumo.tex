\chapter*{Resumo}

\noindent 
Este estudo propõe a aplicação de técnicas de inteligência artificial (IA) no âmbito do mercado financeiro, visando prever e recomendar estrategicamente investimentos. O trabalho inicia com uma revisão histórica dos mercados financeiros, abordando a teoria da eficiência de mercado, que postula que os preços refletem todas as informações disponíveis, destacando ainda a influência das emoções e comportamentos na formação dos preços das ações. Uma investigação sobre a eficiência de mercados emergentes revela oportunidades de investimento devido a desvios sistemáticos nos preços.
A literatura é abordada para destacar a evolução de modelos de análise técnica, fundamentalista, quantitativa e de sentimentos. O estudo segue revisando diversas técnicas de IA aplicadas à previsão de ativos financeiros, incluindo modelos lineares e aqueles fundamentados em IA. A metodologia proposta abrange a extração, geração e seleção de variáveis, culminando na implementação de uma máquina de previsão que combina modelos estatísticos, de classificação e de regressão por meio de técnicas de \textit{ensemble}. 
Os resultados encontrados indicam a viabilidade de alcançar retornos significativos por meio de técnicas de previsão, sendo a máquina de previsão comparada favoravelmente com a técnica de \textit{buy and hold}. Essa abordagem integrativa revela-se promissora para orientar decisões financeiras mais informadas e estratégicas no mercado.
\vfill%
\noindent Palavras-chave: Previsão de ações; Mercado financeiro; Inteligência artificial; Séries temporais; Análise de dados financeiros; Ensemble.
