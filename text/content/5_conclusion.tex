\chapter{Conclusão}
\label{cap:conclusao}
O cerne deste trabalho consistiu na aplicação de técnicas de inteligência artificial no contexto do mercado financeiro. Para alcançar esse objetivo, a pesquisa teve início com a apresentação dos conceitos fundamentais do mercado financeiro, percorrendo desde sua origem até análises mercadológicas de investimento. Além disso, uma revisão detalhada da literatura foi realizada, explorando as técnicas mais relevantes e contemporâneas para a previsão de ativos financeiros.
Com base nesse embasamento teórico robusto, a proposta central do trabalho envolveu o desenvolvimento de uma máquina de previsão utilizando abordagens de \textit{ensemble}. Essa máquina é alimentada com diversas métricas econômicas e tem a capacidade de fornecer a probabilidade de um determinado ativo financeiro apresentar movimento de alta ou baixa. Posteriormente, conduziu-se um processo de recomendação, visando maximizar os ganhos em cada operação, integrando assim efetivamente a análise preditiva com a tomada de decisões estratégicas.

Foram conduzidos experimentos com uma variedade de dez algoritmos de previsão, distribuídos em três categorias principais: classificação, representada pelos modelos \ac{KNN}, \ac{SVM} e \ac{LR}; modelos estatísticos, incluindo \ac{ARIMA}, \ac{SARIMA} e \ac{GARCH}; e modelos de regressão, como \ac{MLP}, \ac{SVR} e \ac{RF}. Os resultados de cada categoria foram combinados usando uma abordagem de \textit{ensemble}, e, por fim, um modelo de \ac{MLP} foi empregado na saída da máquina de previsão.
Os resultados desses algoritmos indicaram claramente a viabilidade de alcançar lucros substanciais. No entanto, é notável que a máquina de previsão proposta não se destacou como a melhor em todos os testes; no entanto, obteve resultados significativos que superaram a estratégia de \textit{buy and hold} em 83,33\% das instâncias. Um achado interessante é que os modelos que alcançaram a melhor acurácia e \ac{F1} não necessariamente alcançaram os melhores resultados na fase de recomendação de investimento, sugerindo que essas métricas podem não ser as mais apropriadas para o cenário específico de investimentos. Essa observação destaca a necessidade de considerar métricas mais alinhadas com os objetivos práticos de maximização de ganhos durante a tomada de decisões de investimento.

Para futuras melhorias deste trabalho, recomenda-se refinamentos no cálculo do erro no modelo de saída (\ac{MLP} OUT), atribuindo maior peso às classificações corretas mais impactantes, especialmente aquelas associadas a maiores variações. Outra abordagem a ser considerada seria a substituição de modelos que não alcançaram resultados satisfatórios durante os experimentos, visando aprimorar a eficácia da máquina de previsão.
Além disso, uma extensão interessante seria a exploração de uma classe de modelos evolutivos para avaliar o impacto do aprendizado contínuo nesse contexto específico. A incorporação de abordagens evolutivas poderia proporcionar insights valiosos sobre a adaptabilidade da máquina de previsão em face de mudanças dinâmicas no mercado financeiro.

Todo o código utilizado para realizar esses experimentos está disponível no Github\footnote{https://github.com/Sr-Souza-dev} em um repositório denominado "TCC-AI". Essa disponibilidade facilita a replicação dos experimentos, permitindo a análise detalhada e a validação dos resultados por outros pesquisadores e profissionais interessados no campo da previsão financeira utilizando técnicas de inteligência artificial.

\section*{Apêndice A - Tabela de Otimização}
\label{ap:tabelas}
\LTXtable{\textwidth}{tabelas/hyperparam}



