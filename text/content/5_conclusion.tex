\chapter{Conclusão}
\label{cap:conclusao}
O objetivo principal deste trabalho experimental foi aplicar técnicas de inteligência artificial no contexto do mercado financeiro. Para alcançar esse objetivo, o estudo iniciou-se com a apresentação dos conceitos fundamentais do mercado financeiro, abrangendo desde a sua origem até as análises mercadológicas de investimento. Além disso, foi conduzida uma revisão da literatura, explorando as técnicas mais relevantes e contemporâneas para previsão de ativos financeiros.
Com base nesse embasamento teórico, a proposta central do trabalho consistiu no desenvolvimento de uma máquina de previsão utilizando abordagens de \textit{Ensemble}. Essa máquina recebe como entrada diversas métricas econômicas e é capaz de fornecer a probabilidade de um determinado ativo financeiro apresentar um movimento de alta ou baixa. Em seguida, foi realizado um processo de recomendação visando maximizar os ganhos em cada operação.

Foram realizados experimentos com três algoritmos de previsão pertencentes à classe de classificação, nomeadamente \ac{KNN}, \ac{SVM} e \ac{LR}, cujos resultados foram combinados por meio de uma técnica de \textit{ensemble}. Os resultados obtidos por esses algoritmos demonstraram fortemente a viabilidade de alcançar lucros significativos. Nesse contexto, o modelo \ac{KNN} obteve os melhores desempenhos nos experimentos, superando consideravelmente a estratégia de \textit{buy and hold}. O modelo de \textit{Ensemble} proposto também apresentou resultados superiores a essa estratégia, embora de forma menos expressiva em comparação com o \ac{KNN}. Por outro lado, o modelo \ac{SVM} obteve resultados equivalentes ao \textit{buy and hold}, enquanto o modelo LR ficou abaixo dessa métrica.

A fim de aprimorar os resultados e concluir a abordagem apresentada no Capítulo \ref{cap:abordagem_proposta}, propõe-se as seguintes etapas de continuidade:

\begin{itemize}
    \item implementar/testar uma variedade de modelos de previsão, abrangendo as três classes de algoritmos da máquina de previsão proposta, incluindo modelos estatísticos, de classificação e de regressão, respectivamente;

    \item realizar o processo de otimização dos hiperparâmetros dos modelos implementados, a fim de buscar as melhores configurações para cada um deles;

    \item avaliar e selecionar os melhores modelos de previsão, levando em consideração cada classe de algoritmo utilizada;

    \item implementar e avaliar diversos algoritmos de previsão como último preditor na máquina de previsão proposta;

    \item implementar e aplicar a estratégia de investimento proposta, utilizando como base a saída de cada classe de algoritmos e também a saída da máquina de previsão como um todo. Essa estratégia buscou aproveitar as previsões geradas pelos modelos estatísticos, de classificação e regressão, bem como a combinação deles realizada pela máquina de previsão, para tomar decisões de compra e venda de ativos financeiros;

    \item avaliar o desempenho de cada algoritmo individualmente e em conjunto com a máquina de previsão, utilizando métricas estatísticas e econométricas como base de análise.
\end{itemize}

Para garantir o cumprimento das etapas necessárias para a conclusão deste trabalho, foi elaborado um cronograma detalhado, apresentado na Tabela \ref{tab:cronoAtv}. Esse cronograma foi projetado para orientar o desenvolvimento do trabalho em termos de prazos e permitir um acompanhamento efetivo do progresso. Cada etapa do processo foi atribuída a um período específico, levando em consideração sua complexidade e duração estimada. 


\begin{table}    
    \centering
    \begin{tabular}{|l|c|c|c|c|c|} \hline
        & Julho & Agosto & Setembro & Outubro & Novembro \\ \hline
        Implementar/testar preditores & X & & & & \\ \hline
        Otimizar hiperparâmetros & X & X & & & \\ \hline
        Avaliar e selecionar os melhores preditores & & X & & & \\ \hline
        Finalizar a máquina de previsão & & & X & X & \\ \hline
        Implementar a estratégia de investimento & & & & X &\\ \hline
        Avaliar o desempenho individual e geral & & & & & X \\ \hline
    \end{tabular}
\caption{Cronograma de atividades.}
\label{tab:cronoAtv}
\end{table} 





